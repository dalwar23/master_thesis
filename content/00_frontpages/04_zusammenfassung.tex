\chapter*{Zusammenfassung}
\label{cha:zusammenfassung}

Das Verstehen von Gemeinschaftsstrukturen in einem Diagramm gibt einen Einblick in die grundlegenden Eigenschaften eines Netzwerks, indem die Eigenschaften und die Beziehung zwischen den Knoten beobachtet werden. Diese Dissertation wird die State-of-the-Art-Community-Erkennungsalgorithmen für große Netzwerke untersuchen, vorzugsweise im Blockchain / Distributed-Ledger-Bereich. Mit der wachsenden Beliebtheit von Community Detection und Blockchain wächst das Interesse an der jeweiligen Domain gleichzeitig. Das Erkennen von Communities in großen Netzwerken ist aufgrund der Komplexität von Zeit und Raum des zugrunde liegenden Algorithmus eine Herausforderung, und um Änderungen im Netzwerk zu beobachten, sind zusätzliche Ansätze erforderlich. Diese Arbeit schlägt einen prototypischen Rahmen vor, um Community-Strukturen in Blockchain-Daten aufzuspüren und danach Veränderungen in Communities zu beobachten. Alle Umsetzungsschritte des Rahmens sind klar definiert. Es bewertet das Framework in Bezug auf Zeit- und Raumkomplexität mit verschiedenen bekannten Community-Erkennungsalgorithmen. Es werden außerdem zusätzliche Schritte zum Beobachten von Änderungen in der Community zu einem bestimmten Zeitstempel vorgeschlagen. Dieses Framework kann leicht modifiziert werden, um Änderungen in einem Blockchain-Netzwerk zu beobachten. Eine signifikante Verbesserung dieses prototypischen Frameworks kann bei der Verarbeitung von Zeitstempeldatensätzen unter Verwendung von High-End-Systemen, paralleler oder verteilter Datenverarbeitung erreicht werden.