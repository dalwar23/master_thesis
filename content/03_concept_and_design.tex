\chapter{Concept and Design}\label{cha:3_concept_and_design}
In chapter (\ref{cha:2_relatedwork}), section (\ref{sec:dynamic_community_algorithms}) a few techniques for dynamic community detection and observation over time had been discussed briefly. All those methods of observing community evaluation have it's own way of dealing with the challenges faced during community detection and observation. This chapter of the thesis will propose and explore a prototypical framework for detecting and observing community over time focusing on large-scale transaction-based networks, specifically on blockchain transaction networks.

\section{Proposed Framework Architecture}\label{sec:framework}
This thesis proposes and introduces a new community detection and observation framework called NEOChain (\underline{N}etwork \underline{E}volution \underline{O}bservation for Block\underline{chain}). It is a prototypical framework for community detection and observation in large-scale networks, preferablly in blockcahin transaction networks domain. Blockchain transaction data is public and the size of the network is massive. As discussed in chapter (\ref{cha:2_relatedwork}), blockchain networks are a suitable match for the scope of this thesis. The proposed prototypical framework can be expressed in brief as follwing:

\begin{enumerate}[label=(\roman*)]
\item Find top $n$ communities (hence forth, \textit{target communities}) $C_t^{top}$ from a snapshot of a graph $\mathcal{G}_t$ at time $t$
\item
\begin{enumerate}
	\item Generate a sub-snapshot $\mathcal{G}_{sub_{t}}$ of graph $\mathcal{G}_t$, where vertex $V_t 		\in \mathcal{G}_{sub_{t}}$ is like $V_t \in C_{top}$ 
	\item Create a graph $\mathcal{G}_{t, t+1}$ by merging sub-snapshot $\mathcal{G}_{sub_{t}}$ with 		new graph snapshot $\mathcal{G}_{t+1}$ at time $t+1$
\end{enumerate}
\item Find top $n$ communities $C_{t+1}^{top}$ from snapshot of graph $\mathcal{G}_{t, t+1}$ at time $t+1$
\item Find maximum overlapping communities $C_{t, t+1}^{max}$ from $C_t^{top}$ and $C_{t+1}^{top}$ and report cluster's status in community structure life cycle.
\end{enumerate}

Phase (i) of the framework takes in a very large, weighted and undirected graph as input and finds top $n$ communities in that graph. For the scope of the thesis, this framework is focused on massive blockchain transaction network data represented in graphs. Selection of top $n$ communities is done by the size of the communities detected. The total number of nodes in a community represents the community size. A set of top $n$ communities is returned at the end of phase (i). For example, let's consider that 20 [$C_1, C_2, C_3, C_4, \ldots , C_{19}, C_{20}$] communities were detected at phase (i) and at time $t$, $n_{c} = 4$. The framework will select top $4$ communities by analyzing sizes of the communities and select top $4$ largest communities and return the set of top communities at time  $t$ as $C_t^{top}$. 

Blockchain transactions are massive in numbers, so to get a grip on the transaction data to detect communities and observe the evolution of communities, it is better to set up particular properties like the number of communities to observe before starting to wrangle blockchain transaction data. It also helps the system to detect better evolution on those top $n$ communities down the line. It reduces the number of \textit{edges} in a graph that is not a part of a considerable community and also discards the possibility of having self links in evolution of the communities. Targeting top $n$ communities also allows to detect communities in medium to high end systems depending on how many nodes and edges are being computed. A careful and strategic approach to select the number of target communities can reduce the run-time and lower the memory consumption.

Community structure has many different scenarios in a community life cycle (\ref{fig:community-evolution}). For observing any kind of evolution there must be a reference point from where the calculation are being drawn to observe how the communities have changed over time. If communities are being detected in different timestamps without a reference point or change point, there might be loss of critical information and the community evolution history might be lost. In practice, it is necessary to know the previous state to understand what has changed in the current state. In phase (ii)(a), to avoid loss of information and to preserve the community evolution history, this framework creates a change point sub-graph from the original graph to be returned as a sub-graph at time $t$. This particular sub-graph $\mathcal{G}_{sub_{t}}$ contains all the nodes and edges that are present in the top $n$ communities from phase (i). This will help the framework to reduce the number of nodes and edges being computed in the next phase without losing any information for the target communities.

In phase (ii)(b), framework will merge two snapshot and create a merged snapshot of the graph. In this phase, the framework will merge the subgraph $\mathcal{G}_{sub_{t}}$, from time $t$ and the new graph snapshot $\mathcal{G}_{t+1}$ from time $t+1$ resulting in a graph $\mathcal{G}_{t, t+1}$, which will contain all the nodes from target communities. This process is described in details in section (\ref{sec:dynamic_community_algorithms}). At this stage of the process, the framework uses a similar process like Clique Percolation Method (\ref{sec:dynamic_community_algorithms}) and \cite{ref-23}. In this process, only the nodes and edges corresponding to the target communities are merged not all the nodes and edges from time $t$.

Once the merge is complete, in phase (iii) the framework will apply the same algorithm from phase (i) on graph $\mathcal{G}_{t, t+1}$ at  time $t+1$ for detecting communities resulting in $C_{t+1}^{top}$. This way, the resulting community set will have the same number of communities as phase (i).

In phase (iv), the framework computes the relative overlap between two sets ($C_t^{top}$ and $C_{t+1}^{top}$) of communities each having $n$ comminities. Calculating the relative overlap gives the insight of changes in the communities from time $t$ to time $t+1$. To calculate the relative overlap, a process is described in \cite{ref-23} intorduced by Palla et al. in 2007. The image of any community in $C_{t+1}^{top}$ at time $t$ is the community of $C_t^{top}$ that has the largest relative ovelap with it. Pall et al. introduced that if two patitions $\mathcal{C}_x$ and $\mathcal{C}_y$ of a graph are similar, each cluster of $\mathcal{C}_x$ will be very similar to one cluster pf $\mathcal{C}_y$, and vice versa. It is very important to identify the pairs of corresponding clusters. For instance, if the information about time evolution of a graph is available, it is possible to moitor the dynamics of single cluster by keeping track of each cluster at different time stamps \cite{ref-23}. For cluster $\mathcal{C}_{x_{i}}$ and $\mathcal{C}_{y_{j}}$, their similarity can be defined through the \textit{relative overlap} $S_{ij}$

\begin{equation}\label{eq:relative_overlap}
S_{ij} = \dfrac{\lvert C_{x_{i}} \cap C_{y_{j}} \rvert}{\lvert C_{x_{i}} \cup C_{y_{j}} \rvert}
\end{equation} 

For the scope of this thesis and for the proposed prototypical community observation frame work, equation (\ref{eq:relative_overlap}) (see also \ref{eq:neihborhood-overlap}) can be modified as following:

\begin{equation}
\mathcal{C}(t) = \dfrac{\lvert \mathcal{C}(t) \cap \mathcal{C}(t+1) \rvert}{\lvert \mathcal{C}(t) \cup \mathcal{C}(t+1) \rvert}
\end{equation}

Time evolution of a community $\mathcal{C}$ can be described by the means of relative overlap $\mathcal{C}(t)$ between states of community separated by a single time step.

Each step of this framework has been discussed in details with reasoning in the next chapter (\ref{cha:4_implementaiton_and_evaluation}). Also the run-time complexity and memory consumption of this prototypical framework has been put to test in the evaluation (\ref{sec:evaluation}) section of this thesis.